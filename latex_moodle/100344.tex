\begin{truefalse}{q-344}
<b>Énoncé</b> : déterminer le domaine de définition de $\sqrt{\dfrac{x-2}{x-3}}$.<br> <b>Solution rédigée à évaluer :</b><br>  « Soit $x\in\mathbb{R}$. L'expression $\dfrac{x-2}{x-3}$ est bien définie ssi $x\neq 3$.<br> Si c'est le cas, l'expression $\sqrt{\dfrac{x-2}{x-3}}$ est bien définie ssi $\dfrac{x-2}{x-3}$ est positive, autrement dit ssi $x-2\geq x-3$ autrement dit jamais. L'expression $\sqrt{\dfrac{x-2}{x-3}}$ n'est donc jamais bien définie.»
\item Vrai
\item* Faux
\end{truefalse}

